\documentclass[11pt,a4paper,oneside,english,italian]{book} 

\usepackage[a4paper]{geometry}
\geometry{verbose, tmargin=4cm, bmargin=4cm, lmargin=3cm, rmargin=3cm, headsep=0.5cm, footskip=1cm}

\usepackage[italian]{babel}

% Pacchetto necessario ad impostare l'interlinea
\usepackage{setspace}

% Package impiegato per editare e formattare opportunamente frammenti di pseudocodice.
\usepackage{algorithmic}

% algorithm serve per incapsulare algorithmic al fine di ottenere un oggetto flotable come una figura o una tabella.
\usepackage{algorithm}

% Pacchetto graphicx per inserire figure in LaTeX. Maggiori dettagli sull'utilizzo del pacchetto sono riportati di seguito.
\usepackage{graphicx} 
\graphicspath{{./figure/}}


% Questi pacchetti consentono di centrare il contenuto di una tabella
\usepackage{multirow}
\usepackage{comment}
\usepackage{array}
\newcolumntype{P}[1]{>{\centering\arraybackslash}p{#1}}
\newcolumntype{M}[1]{>{\centering\arraybackslash}m{#1}}
\newcolumntype{L}[1]{>{\arraybackslash}m{#1}}
\usepackage{xcolor,colortbl}
\usepackage{tabularray}
\definecolor{red}{rgb}{1,0,0}

% Metadati
\usepackage{hyperref}  
\hypersetup{
    pdftitle={Progetto di Sistemi Cyberfisici di Mattia Marino},
    pdfauthor={Mattia Marino}
}

% Matematica
\usepackage{amsmath, amsfonts, amssymb, amsthm}

% Indentazione automatica anche nel primo paragrafo
\usepackage{indentfirst}

% Redifinizione della riga di testa delle pagine:
\usepackage{fancyhdr}
\pagestyle{fancy}

\renewcommand{\chaptermark}[1]{\markboth{#1}{}}
\renewcommand{\sectionmark}[1]{\markright{\thesection\ #1}}
\fancyhf{}
\fancyhead[LO]{\bfseries\rightmark}
\renewcommand{\headrulewidth}{0.5pt}
\renewcommand{\footrulewidth}{0.5pt}
\setlength{\headheight}{14.5pt}

% Pacchetti addizionali per inserire più tabelle o figure fianco a fianco
\usepackage{caption}
\usepackage{subcaption}
\usepackage{booktabs}
\usepackage{float}         
	

% Da qui inizia il documento, fino ad ora si è preparato il preambolo
\begin{document}

% Titolo
\frontmatter

% Frontespizio e dedica
\linespread{1}

\thispagestyle{empty}
\large

% INTESTAZIONE DEL FRONTESPIZIO
\begin{center}
	\huge{{\underline{\textsc{UNIVERSIT\`A DEGLI STUDI DEL SANNIO}}}}
\end{center}

\begin{center}   
	\huge{{\textsc{DIPARTIMENTO DI INGEGNERIA}}}
	\\
	\LARGE{\textbf{\\Corso di Laurea Magistrale in \\Ingegneria Informatica}}      
	
\end{center}

%******************************************************************
%                                   Logo Unisannio
%******************************************************************
\begin{figure}[h]
	\begin{center}
		\includegraphics[scale=0.3]{figure/logo/logoUniSannio_new.jpg}
	\end{center}
\end{figure}
%******************************************************************                                

\vspace{0.25cm}

% Titolo
\begin{center}
    Analisi e Controllo di Sistemi Cyberfisici
    
	{\LARGE \textbf{PROGETTAZIONE DI UN SEMAFORO A CORSIA MULTIPLA ORIENTATO AL PARALLELISMO} \smallskip\\}                                               
\end{center}

% Prof. e candidato
\vspace{2cm}
\begin{tabular}{ll}
	\textit{Prof.ssa:}        \hspace{3cm}   	& \textit{Autore:}\\
	\textbf{Elisa Mostacciuolo}      \hspace{5cm}     & \textbf{Mattia Marino}\\   
	\hspace{3cm}      & \textit{Matr:399000634}                                       					   
	%                        
\end{tabular}

% ANNO ACCADEMICO     
\vspace{2.5cm}
\begin{center}
	\textsc{Anno Accademico 2024/2025}
\end{center}

\newpage


% Numerazione delle prime pagine in numeri romani
\pagenumbering{roman}

% Set del conunter al valore 1, questo evita che vengono conteggiate anche la pagine del frontespizio, facendo partire il conteggio da 3 piuttosto che da 1
\setcounter{page}{0}

% Indice della tesi
\tableofcontents

% Rispristino il valore dell'interlinea a 1.5, quadunque nel frontespizio venisse posto ad 1.0
\onehalfspacing

% Introduzione
\input{00 - Intro/01 - Introduzione}
\chapter*{Problem Statement}
\label{cap:problem-statement}

\addcontentsline{toc}{chapter}{Problem Statement}
\lhead{\bfseries PROBLEM STATEMENT}
\rhead{\thepage}

Il progetto ivi trattato si pone l'obiettivo di progettare e modellare il comportamento di un insieme di semafori stradali in un'intersezione. In particolare, si è scelto di riprogettare un semaforo già esistente, con lo scopo di migliorarne l'efficienza e la sicurezza. Il semaforo scelto a tale scopo è quello presente all'incrocio tra Via del Pomerio, Ponte Vanvitelli, Via Posillipo e Corso Vittorio Emanuele III, a Benevento. Al fine di spiegare meglio la dinamica dell'intersezione, si riportano due schermate catturate da Google Maps, che mostrano l'incrocio in questione (Fig. \ref{fig:intersection_top} e Fig. \ref{fig:intersection_3d}).

\begin{figure}[H]
    \centering
    \begin{minipage}[b]{.45\textwidth}
        \centering
        \includegraphics[width=\textwidth]{figure/intersection/incrocio_top.png}
        \caption{Vista dall'alto dell'intersezione}
        \label{fig:intersection_top}
    \end{minipage}
    \hfill
    \begin{minipage}[b]{.45\textwidth}
        \centering
        \includegraphics[width=\textwidth]{figure/intersection/incrocio_3d.png}
        \caption{Vista 3D dell'intersezione}
        \label{fig:intersection_3d}
    \end{minipage}
\end{figure}

Come si può quindi dedurre dall'immagine, a Est si trovano tre semafori che consentono di andare a Nord, Sud e Ovest. A Ovest, invece, si trovano due semafori che consentono di andare a Nord e a Sud. Il semaforo a Nord, infine, consente di andare a Ovest o a Sud. Ogni combinazione provenienza-destinazione si trova su un'apposita corsia e ha un semaforo dedicato.

Al fine di modellare il comportamento del semaforo, si è scelto di utilizzare il formalismo delle Reti di Petri, in quanto permette di modellare sistemi concorrenti e distribuiti in modo chiaro e intuitivo. Nel caso modellato, per migliorare l'efficienza del semaforo, si è scelto di permettere il passaggio di più veicoli contemporaneamente, qualora possibile, imponendo vincoli per evitare che due o più veicoli si incrocino. Ad esempio è consentito il passaggio contemporaneo di un veicolo proveniente da Est diretto verso Nord e uno proveniente da Ovest e diretto verso Sud. Al contrario, invece, non è possibile attraversare contemporaneamente l'incrocio con due veicoli provenienti da direzioni diverse e diretti nella stessa direzione, in quanto ciò potrebbe causare un incidente. Inoltre, non è possibile che due veicoli si incontrino all'interno dell'incrocio, come ad esempio un veicolo che attraversa l'incrocio da Est a Ovest e un veicolo che attraversa l'incrocio da Nord a Sud.

Tali vincoli sono stati modellati con l'utilizzo di GMEC, come mostrato nelle sezioni successive.

% Corpo del documento
\mainmatter
\chapter{Reti di Petri}
\label{cap:cap1}
\lhead{\textbf{\rightmark}}
In questo capitolo viene discussa la teoria che si trova alla base della Rete di Petri utilizzata per modellare questo progetto. Si parte dal dare una definizione generale, per poi andare più nel dettaglio spiegando i concetti di matrice di pre e post incidenza, marcatura, e GMEC.
\newpage

\section{Definizione}
\label{sec:1.1}
Una Rete di Petri è un formalismo matematico per la modellazione di sistemi distribuiti discreti. In termini più semplici, è un modo per descrivere come funzionano i sistemi in cui accadono eventi discreti e concorrenti, ovvero eventi che possono accadere in parallelo. Immagina un sistema con diverse attività che possono svolgersi contemporaneamente o in sequenza, con risorse che vengono utilizzate e rilasciate: una Rete di Petri fornisce un linguaggio grafico e matematico per rappresentare e analizzare questo tipo di sistemi.

\subsection{Componenti Chiave di una Rete di Petri}
Una Rete di Petri è rappresentata graficamente come un grafo bipartito composto da quattro elementi fondamentali:
\begin{itemize}
    \item \textbf{Posti:} Rappresentati da cerchi, i posti rappresentano le condizioni o gli stati del sistema. Ad esempio, un posto potrebbe rappresentare la disponibilità di una risorsa, lo stato di un processo (attivo, inattivo, in attesa), o una condizione logica (vero/falso).
    \item \textbf{Transizioni:} Rappresentate da rettangoli o barre, le transizioni rappresentano gli eventi o le azioni che possono accadere nel sistema. L'attivazione di una transizione causa un cambiamento nello stato del sistema.
    \item \textbf{Archi:} Rappresentati da frecce dirette, gli archi connettono i posti alle transizioni e le transizioni ai posti, definendo le relazioni di causa-effetto tra stati ed eventi. Un arco che va da un posto a una transizione indica una pre-condizione per l'attivazione della transizione; un arco che va da una transizione a un posto indica un effetto post-attivazione della transizione.
    \item \textbf{Marcatura:} La marcatura di una rete è la distribuzione di token (generalmente rappresentati da punti neri) all'interno dei posti. La marcatura rappresenta lo stato corrente del sistema. Il numero di token in un posto indica quante "unità" di quella condizione sono presenti.
\end{itemize}

\subsection{Funzionamento di una Rete di Petri}
L'evoluzione di una Rete di Petri è regolata da una semplice regola di scatto (firing):
\begin{itemize}
    \item \textbf{Abilitazione:} Una transizione è abilitata (può scattare) se tutti i posti di input (i posti da cui partono archi verso la transizione) contengono almeno un token.
    \item \textbf{Scatto:} Quando una transizione abilitata scatta, vengono rimossi un token da ciascun posto di input e viene aggiunto un token a ciascun posto di output (i posti a cui arrivano archi dalla transizione).
\end{itemize}

Nelle Reti di Petri, l'enfasi è posta sulle regole che governano le transizioni tra i vari stati, piuttosto che sugli stati stessi. Questo approccio rende il modello più versatile e adatto a rappresentare sistemi dinamici e complessi, dove il numero di stati possibili può essere indefinito o teoricamente infinito.

\newpage

\section{Matrici di Pre e Post incidenza}
\label{sec:1.2}
Le matrici di pre e post-incidenza, insieme alla matrice di incidenza, sono strumenti matematici fondamentali per rappresentare e analizzare le Reti di Petri in modo formale. Permettono di descrivere le connessioni tra posti e transizioni e di studiare il comportamento dinamico della rete attraverso calcoli matriciali.

\subsection{Matrice di Pre-incidenza}
La matrice di pre-incidenza, indicata con \textit{Pre}, descrive le connessioni che vanno dai posti alle transizioni. È una matrice \textit{n x m}, dove \textit{n} è il numero di posti e \textit{m} è il numero di transizioni nella rete. L'elemento \textit{Pre(p, t)} della matrice indica il numero di archi che vanno dal posto \textit{p} alla transizione \textit{t}. In altre parole, rappresenta il numero di token che devono essere presenti nel posto p affinché la transizione t sia abilitata a scattare.

\begin{itemize}
    \item Se \textit{Pre(p, t) = 0}, non esiste un arco che connette il posto \textit{p} alla transizione \textit{t}.
    \item Se \textit{Pre(p, t) = 1}, esiste un singolo arco che connette il posto \textit{p} alla transizione \textit{t}.
    \item Se \textit{Pre(p, t) > 1}, esistono più archi (archi multipli) che connettono il posto \textit{p} alla transizione \textit{t}.
\end{itemize}

\subsection{Matrice di Post-incidenza}
La matrice di post-incidenza, indicata con \textit{Post}, descrive le connessioni che vanno dalle transizioni ai posti. È anch'essa una matrice \textit{n x m}, con \textit{n} posti e \textit{m} transizioni. L'elemento \textit{Post(p, t)} indica il numero di archi che vanno dalla transizione \textit{t} al posto \textit{p}. Rappresenta il numero di token che vengono depositati nel posto \textit{p} quando la transizione \textit{t} scatta.

\begin{itemize}
    \item Se \textit{Post(p, t) = 0}, non esiste un arco che connette la transizione \textit{t} al posto \textit{p}.
    \item Se \textit{Post(p, t) = 1}, esiste un singolo arco che connette la transizione \textit{t} al posto \textit{p}.
    \item Se \textit{Post(p, t) > 1}, esistono più archi che connettono la transizione \textit{t} al posto \textit{p}.
\end{itemize}

\subsection{Matrice di Incidenza}
La matrice di incidenza, indicata con \textit{C}, è derivata dalle matrici di pre e post-incidenza e fornisce una rappresentazione compatta delle variazioni di marcatura causate dallo scatto delle transizioni. Essa è definita come:
\begin{equation}
    C = Post - Pre
\end{equation}

L'elemento \textit{C(p, t)} della matrice di incidenza indica quindi la variazione netta del numero di token nel posto \textit{p} quando la transizione \textit{t} scatta:

\begin{itemize}
    \item Se \textit{C(p, t) > 0}, lo scatto di \textit{t} aggiunge token al posto \textit{p}.
    \item Se \textit{C(p, t) < 0}, lo scatto di \textit{t} rimuove token dal posto \textit{p}.
    \item Se \textit{C(p, t) = 0}, lo scatto di \textit{t} non influenza il numero di token nel posto \textit{p}.
\end{itemize}
\newpage

\section{Marcatura}
\label{sec:1.3}
La marcatura è un altro concetto centrale nelle Reti di Petri, in quanto rappresenta lo stato del sistema modellato in un dato istante. In termini semplici, la marcatura indica la distribuzione dei \textit{token} (generalmente rappresentati da punti neri) all'interno dei \textit{posti} della rete.

Formalmente, una marcatura \textit{M} di una Rete di Petri è una funzione che associa ad ogni posto \textit{p} un numero intero non negativo, \textit{M(p)}. Questo numero rappresenta il numero di token presenti nel posto \textit{p}. Possiamo rappresentare la marcatura come un vettore, dove l'i-esimo elemento del vettore corrisponde al numero di token nel posto i-esimo.

Graficamente, i token sono disegnati all'interno dei cerchi che rappresentano i posti. Se un posto non contiene token, il suo cerchio è vuoto. Se un posto contiene più di un token, il numero di token è solitamente indicato all'interno del cerchio. Generalmente il numero di token in un posto è rappresentato disegnando un cerchio con un numero di punti neri uguale al numero di token presenti, ma in caso in cui tale numero sia elevato è anche possibile scrivere semplicemente il numero.

La marcatura iniziale, spesso indicata con \textit{M0}, rappresenta lo stato di partenza del sistema. È la marcatura da cui inizia l'esecuzione della rete. La scelta della marcatura iniziale è cruciale, poiché determina il comportamento che la rete può esibire. Come spiegato nella sezione \ref{sec:1.4} è anche possibile che la marcatura iniziale scelta sia illegale per via dei vincoli imposti dalla rete.
\newpage

\section{GMEC}
\label{sec:1.4}
Le Generalized Mutual Exclusion Constraints (GMEC), o Vincoli Generalizzati di Mutua Esclusione, sono un potente formalismo utilizzato nell'analisi e nel controllo delle Reti di Petri, specialmente in contesti dove è necessario garantire che certe combinazioni di marcature non si verifichino mai. In altre parole, le GMEC definiscono insiemi di posti la cui somma di token non deve superare una determinata soglia. Questo è particolarmente utile per modellare risorse condivise, vincoli di capacità, o condizioni di sicurezza in sistemi concorrenti.

Una GMEC è espressa tipicamente come una disuguaglianza lineare sulla marcatura \textit{M} della Rete di Petri:

\begin{equation}
    \sum_{p \in P} w_{p} \cdot M(p) \leq k
\end{equation}

dove:

\begin{itemize}
    \item $P$ è un insieme di posti della rete.
    \item $w_{p}$ è un peso non negativo associato al posto p.
    \item $M(p)$ è il numero di token nel posto p nella marcatura M.
    \item $k$ è una costante intera non negativa che rappresenta la capacità o la soglia massima.
\end{itemize}

Volendo scrivere il tutto in forma vettoriale e definendo il vettore $w$ come il vettore dei pesi associati ai posti, e il vettore $M$ come il vettore delle marcature, possiamo scrivere la disuguaglianza come:

\begin{equation}
    w^{T} \times M \leq k
\end{equation}

\subsection{Posto monitor}
Il concetto di "posto monitor", o "supervisore", è strettamente legato all'implementazione delle GMEC nelle Reti di Petri. L'obiettivo è di trasformare una disuguaglianza (la GMEC) in un'uguaglianza, introducendo un nuovo posto, il posto monitor, che "monitora" il rispetto del vincolo.

Data una GMEC scritta in forma vettoriale:

\begin{equation}
    w^{T} \times M \leq k
\end{equation}

Introduciamo un nuovo posto, che chiameremo $p_{m}$ (posto monitor), che avrà come matrice di incidenza un vettore riga  di lunghezza \textit{n} (dove \textit{n} è il numero di transizioni della rete):

\begin{equation}
    C_{m} = -w^{T} \times C
\end{equation}

Da tale vettore è possibile quindi capire come il posto monitor si colleghi al resto della rete già esistente. Inoltre, per calcolare la marcatura iniziale del posto monitor:

\begin{equation}
    M(p_{m}) = k - w^{T} \times M_{0}
\end{equation}

\subsection{GMEC multiple}
Qualora ci fossero GMEC multiple, come nel caso ivi preso in esame, è possibile calcolare rapidamente le matrici di incidenza dei posti monitor associati a ciascuna GMEC e le relative marcature iniziali utilizzando delle semplici operazioni matriciali.

Considerando quindi un inzieme di GMEC, è possibile esprimere i loro vettori peso come righe di una matrice $W$:

\begin{equation}
    W = \begin{bmatrix}
        w_{1} \\
        w_{2} \\
        \vdots \\
        w_{m}
    \end{bmatrix}
\end{equation}

L'equazione quindi diventa:

\begin{equation}
    C_{m} = -W \times C
\end{equation}

Similmente, esprimendo le costanti $k$ come un vettore colonna $K$, è possibile calcolare le marcature iniziali dei posti monitor come:

\begin{equation}
    M(p_{m}) = K - W \times M_{0}
\end{equation}
\newpage


\backmatter

\nocite{*}

\end{document}

La marcatura è un altro concetto centrale nelle Reti di Petri, in quanto rappresenta lo stato del sistema modellato in un dato istante. In termini semplici, la marcatura indica la distribuzione dei \textit{token} (generalmente rappresentati da punti neri) all'interno dei \textit{posti} della rete.

Formalmente, una marcatura \textit{M} di una Rete di Petri è una funzione che associa ad ogni posto \textit{p} un numero intero non negativo, \textit{M(p)}. Questo numero rappresenta il numero di token presenti nel posto \textit{p}. Possiamo rappresentare la marcatura come un vettore, dove l'i-esimo elemento del vettore corrisponde al numero di token nel posto i-esimo.

Graficamente, i token sono disegnati all'interno dei cerchi che rappresentano i posti. Se un posto non contiene token, il suo cerchio è vuoto. Se un posto contiene più di un token, il numero di token è solitamente indicato all'interno del cerchio. Generalmente il numero di token in un posto è rappresentato disegnando un cerchio con un numero di punti neri uguale al numero di token presenti, ma in caso in cui tale numero sia elevato è anche possibile scrivere semplicemente il numero.

\subsection{Marcatura Iniziale}
La marcatura iniziale, spesso indicata con $M_{0}$, rappresenta lo stato di partenza del sistema. È la marcatura da cui inizia l'esecuzione della rete. La scelta della marcatura iniziale è cruciale, poiché determina il comportamento che la rete può esibire. Come spiegato nella sezione \ref{sec:1.4} è anche possibile che la marcatura iniziale scelta sia illegale per via dei vincoli imposti dalla rete.

\subsection{Evoluzione della Marcatura}
L'evoluzione di una Rete di Petri è determinata dallo scatto delle transizioni. Lo scatto di una transizione modifica la marcatura della rete secondo le seguenti regole:

\begin{itemize}
    \item \textbf{Abilitazione:} Una transizione \textit{t} è abilitata se ogni posto di input di \textit{t} (cioè ogni posto da cui parte un arco verso \textit{t}) contiene almeno un numero di token pari al peso dell'arco (se non specificato, il peso è 1).
    \item \textbf{Scatto:} Quando una transizione t abilitata scatta: 
    \begin{itemize}
        \item Viene rimosso da ogni posto di input di \textit{t} un numero di token pari al peso dell'arco che lo connette a \textit{t}.
        \item Viene aggiunto a ogni posto di output di \textit{t} (cioè ogni posto a cui arriva un arco da \textit{t}) un numero di token pari al peso dell'arco che lo connette a \textit{t}.
    \end{itemize}
\end{itemize}

Questo processo di scatto delle transizioni causa una transizione da una marcatura all'altra, descrivendo l'evoluzione dinamica del sistema modellato.

\subsection{Raggiungibilità}
Un concetto importante legato alla marcatura è la \textit{raggiungibilità}. Una marcatura \textit{M'} è raggiungibile da una marcatura \textit{M} se esiste una sequenza di scatti di transizioni che, partendo da \textit{M}, porta alla marcatura \textit{M'}. L'insieme di tutte le marcature raggiungibili da una marcatura iniziale $M_{0}$ è detto \textit{insieme di raggiungibilità di $M_{0}$}. L'analisi dell'insieme di raggiungibilità è fondamentale per studiare le proprietà del sistema modellato, come la boundedness (limitazione del numero di token nei posti), la liveness (possibilità di scatto delle transizioni) e la reversibilità (possibilità di tornare alla marcatura iniziale).
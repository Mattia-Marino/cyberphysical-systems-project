Una Rete di Petri è un formalismo matematico per la modellazione di sistemi distribuiti discreti. In termini più semplici, è un modo per descrivere come funzionano i sistemi in cui accadono eventi discreti e concorrenti, ovvero eventi che possono accadere in parallelo. Immagina un sistema con diverse attività che possono svolgersi contemporaneamente o in sequenza, con risorse che vengono utilizzate e rilasciate: una Rete di Petri fornisce un linguaggio grafico e matematico per rappresentare e analizzare questo tipo di sistemi.

\subsection{Componenti Chiave di una Rete di Petri}
Una Rete di Petri è rappresentata graficamente come un grafo bipartito composto da quattro elementi fondamentali:
\begin{itemize}
    \item \textbf{Posti:} Rappresentati da cerchi, i posti rappresentano le condizioni o gli stati del sistema. Ad esempio, un posto potrebbe rappresentare la disponibilità di una risorsa, lo stato di un processo (attivo, inattivo, in attesa), o una condizione logica (vero/falso).
    \item \textbf{Transizioni:} Rappresentate da rettangoli o barre, le transizioni rappresentano gli eventi o le azioni che possono accadere nel sistema. L'attivazione di una transizione causa un cambiamento nello stato del sistema.
    \item \textbf{Archi:} Rappresentati da frecce dirette, gli archi connettono i posti alle transizioni e le transizioni ai posti, definendo le relazioni di causa-effetto tra stati ed eventi. Un arco che va da un posto a una transizione indica una pre-condizione per l'attivazione della transizione; un arco che va da una transizione a un posto indica un effetto post-attivazione della transizione.
    \item \textbf{Marcatura:} La marcatura di una rete è la distribuzione di token (generalmente rappresentati da punti neri) all'interno dei posti. La marcatura rappresenta lo stato corrente del sistema. Il numero di token in un posto indica quante "unità" di quella condizione sono presenti.
\end{itemize}

\subsection{Funzionamento di una Rete di Petri}
L'evoluzione di una Rete di Petri è regolata da una semplice regola di scatto (firing):
\begin{itemize}
    \item \textbf{Abilitazione:} Una transizione è abilitata (può scattare) se tutti i posti di input (i posti da cui partono archi verso la transizione) contengono almeno un token.
    \item \textbf{Scatto:} Quando una transizione abilitata scatta, vengono rimossi un token da ciascun posto di input e viene aggiunto un token a ciascun posto di output (i posti a cui arrivano archi dalla transizione).
\end{itemize}

Nelle Reti di Petri, l'enfasi è posta sulle regole che governano le transizioni tra i vari stati, piuttosto che sugli stati stessi. Questo approccio rende il modello più versatile e adatto a rappresentare sistemi dinamici e complessi, dove il numero di stati possibili può essere indefinito o teoricamente infinito.

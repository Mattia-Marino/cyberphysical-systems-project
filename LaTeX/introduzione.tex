\chapter*{Introduzione}
\label{cap:introduzione}

\addcontentsline{toc}{chapter}{Introduzione}
\lhead{\bfseries INTRODUZIONE}
\rhead{\thepage}

Nel contesto tecnologico attuale, i Cyber-Physical Systems (CPS) stanno 
acquisendo un'importanza sempre maggiore. Questi sistemi integrati, che 
combinano componenti fisici con capacità computazionali, hanno trasformato 
vari settori, dall'industria alla sanità, dai trasporti all'energia. 
Tipicamente, ciò avviene attraverso loop di feedback in cui i processi 
fisici influenzano la computazione e viceversa. Affinché il sistema possa 
rispondere ai cambiamenti esterni e correggere il proprio comportamento, 
è essenziale che la parte fisica e quella computazionale comunichino 
tramite una rete. I CPS possono impiegare sensori, attuatori, elettronica 
e software avanzato che collaborano per monitorare, controllare e 
ottimizzare sistemi reali in tempo reale. Questa interazione tra il 
mondo cibernetico e quello fisico offre numerosi vantaggi, tra cui maggiore 
efficienza operativa, sicurezza migliorata, riduzione dei costi e 
ottimizzazione delle risorse (che nei sistemi embedded sono spesso molto 
limitate). Questo documento descrive e dettaglia il lavoro svolto nella 
progettazione e modellazione di un CPS per il controllo di un sistema 
di semafori stradali in un'intersezione stradale con un focus particolare
sul parallelismo dei veicoli che attraversano qualora possibile.